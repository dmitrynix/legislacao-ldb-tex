\documentclass[]{beamer}
\mode<presentation>

\usepackage[utf8]{inputenc}
\usepackage[portuges]{babel}
\usepackage[T1]{fontenc}
%\usepackage{hyperref}
\usepackage[normalem]{ulem}
\usepackage[]{csquotes}
\usepackage[]{enumerate}

\setbeamertemplate{footline}[page number]
\setbeamercovered{transparent}
\beamertemplatenavigationsymbolsempty

\author[]{
  Dmitry Rocha       \\
  Fellype Sales      \\
  Josué Junior       \\
  Leomar Filho       \\
  Williams Guilherme \\
}

\title{Capítulo IV \\ Da Educação Superior}
\institute{UFPI}
\date{15 de Janeiro de 2015}

\begin{document}
\begin{frame}
  \titlepage
\end{frame}

\begin{frame}
  \huge Art. 43. A educação superior tem por finalidade:
\end{frame}

\begin{frame}{Art. 43. A educação superior tem por finalidade:}
  \only<1>{
    I - Estimular a criação cultural e o desenvolvimento do
    espírito científico e do pensamento reflexivo;
  }
  \only<2>{
    II - formar diplomados nas diferentes áreas de conhecimento,
    aptos para a inserção em setores profissionais e para a participação
    no desenvolvimento da sociedade brasileira, e colaborar na sua
    formação contínua;
  }
  \only<3>{
    III - incentivar o trabalho de pesquisa e investigação científica, visando o
    desenvolvimento da ciência e da tecnologia e da criação e difusão da
    cultura, e, desse modo, desenvolver o entendimento do homem e do meio
    em que vive;
  }
  \only<4>{
    IV - promover a divulgação de conhecimentos culturais, científicos e
    técnicos que constituem patrimônio da humanidade e comunicar o saber
    através do ensino, de publicações ou de outras formas de comunicação;
  }
  \only<5>{
    V - suscitar o desejo permanente de aperfeiçoamento cultural e
    profissional e possibilitar a correspondente concretização, integrando
    os conhecimentos que vão sendo adquiridos numa estrutura intelectual
    sistematizadora do conhecimento de cada geração;
  }
  \only<6>{
    VI - estimular o conhecimento dos problemas do mundo presente, em
    particular os nacionais e regionais, prestar serviços especializados à
    comunidade e estabelecer com esta uma relação de reciprocidade;
  }
  \only<7>{
    VII - promover a extensão, aberta à participação da população, visando à
    difusão das conquistas e benefícios resultantes da criação cultural e da
    pesquisa científica e tecnológica geradas na instituição.
  }
\end{frame}

\begin{frame}
  \huge Art. 44. A educação superior abrangerá os seguintes cursos e programas:
\end{frame}

\begin{frame}{Art. 44. A educação superior abrangerá os seguintes…}
  \only<1>{
    \sout{I - cursos seqüenciais por campo de saber, de diferentes níveis
      de abrangência, abertos a candidatos que atendam aos requisitos
    estabelecidos pelas instituições de ensino;}

    I - cursos seqüenciais por campo de saber, de diferentes níveis
    de abrangência, abertos a candidatos que atendam aos requisitos
    estabelecidos pelas instituições de ensino, desde que tenham concluído o
    ensino médio ou equivalente;
    \footnote{Redação dada pela Lei nº 11.632, de 2007}.
  }
  \only<2>{II - de graduação, abertos a candidatos que tenham concluído o
    ensino médio ou equivalente e tenham sido classificados em processo
  seletivo;}
  \only<3>{III - de pós-graduação, compreendendo programas de mestrado e
    doutorado, cursos de especialização, aperfeiçoamento e outros, abertos a
    candidatos diplomados em cursos de graduação e que atendam às exigências
  das instituições de ensino;}
  \only<4>{IV - de extensão, abertos a candidatos que atendam aos requisitos
  estabelecidos em cada caso pelas instituições de ensino.}
  \only<5>{Parágrafo único. Os resultados do processo seletivo referido no
    inciso II do \textbf{caput} deste artigo serão tornados públicos pelas
    instituições de ensino superior, sendo obrigatória a divulgação da
    relação nominal dos classificados, a respectiva ordem de classificação,
    bem como do cronograma das chamadas para matrícula, de acordo com os
  critérios para preenchimento das vagas constantes do respectivo edital.}
\end{frame}

\begin{frame}
  \huge Art. 45. A educação superior será ministrada em instituições de
  ensino superior, públicas ou privadas, com variados graus de abrangência
  ou especialização.
\end{frame}

\begin{frame}
  \huge Art. 46. A autorização e o reconhecimento de cursos, bem como o
  credenciamento de instituições de educação superior, terão prazos
  limitados, sendo renovados, periodicamente, após processo regular de
  avaliação.
\end{frame}

\begin{frame}{Art. 46. A autorização e o reconhecimento de cursos, …}
  \only<1>{
    § 1º Após um prazo para saneamento de deficiências eventualmente
    identificadas pela avaliação a que se refere este artigo, haverá
    reavaliação, que poderá resultar, conforme o caso, em desativação de
    cursos e habilitações, em intervenção na instituição, em suspensão
    temporária de prerrogativas da autonomia, ou em descredenciamento.
  }

  \only<2>{
    § 2º No caso de instituição pública, o Poder Executivo responsável por sua
    manutenção acompanhará o processo de saneamento e fornecerá recursos
    adicionais, se necessários, para a superação das deficiências.
  }
\end{frame}

\begin{frame}
  \huge Art. 47. Na educação superior, o ano letivo regular, independente do ano
  civil, tem, no mínimo, duzentos dias de trabalho acadêmico efetivo,
  excluído o tempo reservado aos exames finais, quando houver.
\end{frame}

\begin{frame}{Art. 47. Na educação superior, o ano letivo regular, …}
  \only<1>{
    § 1º As instituições informarão aos interessados, antes de cada período
    letivo, os programas dos cursos e demais componentes curriculares, sua
    duração, requisitos, qualificação dos professores, recursos disponíveis e
    critérios de avaliação, obrigando-se a cumprir as respectivas condições.
  }

  \only<2>{
    § 2º Os alunos que tenham extraordinário aproveitamento nos estudos,
    demonstrado por meio de provas e outros instrumentos de avaliação
    específicos, aplicados por banca examinadora especial, poderão ter
    abreviada a duração dos seus cursos, de acordo com as normas dos sistemas
    de ensino.
  }

  \only<3>{
    § 3º É obrigatória a freqüência de alunos e professores, salvo nos
    programas de educação a distância.
  }

  \only<4>{
    § 4º As instituições de educação superior oferecerão, no período noturno,
    cursos de graduação nos mesmos padrões de qualidade mantidos no período
    diurno, sendo obrigatória a oferta noturna nas instituições públicas,
    garantida a necessária previsão orçamentária.
  }
\end{frame}

\begin{frame}
  \huge Art. 48. Os diplomas de cursos superiores reconhecidos, quando registrados, terão validade nacional como prova da formação recebida por seu titular.
\end{frame}

\begin{frame}{Art. 48. Os diplomas de cursos superiores reconhecidos, …}
  \only<1>{
    § 1º Os diplomas expedidos pelas universidades serão por elas próprias
    registrados, e aqueles conferidos por instituições não-universitárias
    serão registrados em universidades indicadas pelo Conselho Nacional de
  Educação.}

  \only<2>{
    § 2º Os diplomas de graduação expedidos por universidades estrangeiras serão
    revalidados por universidades públicas que tenham curso do mesmo nível e área ou
    equivalente, respeitando-se os acordos internacionais de reciprocidade ou
  equiparação.}

  \only<3>{
    § 3º Os diplomas de Mestrado e de Doutorado expedidos por universidades
    estrangeiras só poderão ser reconhecidos por universidades que possuam cursos de
    pós-graduação reconhecidos e avaliados, na mesma área de conhecimento e em nível
  equivalente ou superior.}
\end{frame}

\begin{frame}
  \huge Art. 49. As instituições de educação superior aceitarão a transferência de
  alunos regulares, para cursos afins, na hipótese de existência de vagas, e
  mediante processo seletivo.

  Parágrafo único. As transferências ex officio dar-se-ão na forma da lei.
\end{frame}

\begin{frame}
  \huge Art. 50. As instituições de educação superior, quando da ocorrência de vagas,
  abrirão matrícula nas disciplinas de seus cursos a alunos não regulares que
  demonstrarem capacidade de cursá-las com proveito, mediante processo seletivo
  prévio.
\end{frame}

\begin{frame}
  \huge Art. 51. As instituições de educação superior credenciadas como universidades,
  ao deliberar sobre critérios e normas de seleção e admissão de estudantes,
  levarão em conta os efeitos desses critérios sobre a orientação do ensino
  médio, articulando-se com os órgãos normativos dos sistemas de ensino.
\end{frame}

\begin{frame}
  \huge Art. 52. As universidades são instituições pluridisciplinares de formação
  dos quadros profissionais de nível superior, de pesquisa, de extensão e de
  domínio e cultivo do saber humano, que se caracterizam por:
\end{frame}

\begin{frame}{Art. 52. As universidades são instituições…}
  \only<1>{
    I - produção intelectual institucionalizada mediante o estudo sistemático
    dos temas e problemas mais relevantes, tanto do ponto de vista científico e
    cultural, quanto regional e nacional;
  }

  \only<2>{
    II - um terço do corpo docente, pelo menos, com titulação acadêmica de
    mestrado ou doutorado;
  }

  \only<3>{
    III - um terço do corpo docente em regime de tempo integral.
  }

  \only<4>{
    Parágrafo único. É facultada a criação de universidades especializadas por
    campo do saber.
  }
\end{frame}

\begin{frame}{}
  \huge Art. 53. No exercício de sua autonomia, são asseguradas às universidades, sem
  prejuízo de outras, as seguintes atribuições:
\end{frame}

\begin{frame}{Art. 53. No exercício de sua autonomia, são asseguradas…}
  \only<1>{
    I - criar, organizar e extinguir, em sua sede, cursos e programas de educação
    superior previstos nesta Lei, obedecendo às normas gerais da União e, quando
    for o caso, do respectivo sistema de ensino;
  }

  \only<2>{
    II - fixar os currículos dos seus cursos e programas, observadas as diretrizes
    gerais pertinentes;
  }

  \only<3>{
    III - estabelecer planos, programas e projetos de pesquisa científica,
    produção artística e atividades de extensão;
  }

  \only<4>{
    IV - fixar o número de vagas de acordo com a capacidade institucional e as
    exigências do seu meio;
  }

  \only<5>{
    V - elaborar e reformar os seus estatutos e regimentos em consonância com as
    normas gerais atinentes;
  }

  \only<6>{
    VI - conferir graus, diplomas e outros títulos;
  }

  \only<7>{
    VII - firmar contratos, acordos e convênios;
  }

  \only<8>{
    VIII - aprovar e executar planos, programas e projetos de investimentos
    referentes a obras, serviços e aquisições em geral, bem como administrar
    rendimentos conforme dispositivos institucionais;
  }

  \only<9>{
    IX - administrar os rendimentos e deles dispor na forma prevista no ato de
    constituição, nas leis e nos respectivos estatutos;
  }

  \only<10>{
    X - receber subvenções, doações, heranças, legados e cooperação financeira
    resultante de convênios com entidades públicas e privadas.
  }

  \only<11>{
    Parágrafo único. Para garantir a autonomia didático-científica das
    universidades, caberá aos seus colegiados de ensino e pesquisa decidir, dentro
    dos recursos orçamentários disponíveis, sobre:

    I - criação, expansão, modificação e extinção de cursos;

    II - ampliação e diminuição de vagas;

    III - elaboração da programação dos cursos;

    IV - programação das pesquisas e das atividades de extensão;

    V - contratação e dispensa de professores;

    VI - planos de carreira docente.
  }
\end{frame}

\begin{frame}
  Art. 54. As universidades mantidas pelo Poder Público gozarão, na forma da
  lei, de estatuto jurídico especial para atender às peculiaridades de sua
  estrutura, organização e financiamento pelo Poder Público, assim como dos seus
  planos de carreira e do regime jurídico do seu pessoal.
\end{frame}

\begin{frame}{Art. 54. As universidades mantidas pelo Poder Público gozarão…}
  \only<1>{
    § 1º No exercício da sua autonomia, além das atribuições asseguradas pelo
    artigo anterior, as universidades públicas poderão:
  }

  \only<2>{
    I - propor o seu quadro de pessoal docente, técnico e administrativo, assim
    como um plano de cargos e salários, atendidas as normas gerais pertinentes e
    os recursos disponíveis;
  }

  \only<3>{
    II - elaborar o regulamento de seu pessoal em conformidade com as normas
    gerais concernentes;
  }

  \only<4>{
    III - aprovar e executar planos, programas e projetos de investimentos
    referentes a obras, serviços e aquisições em geral, de acordo com os recursos
    alocados pelo respectivo Poder mantenedor;
  }

  \only<5>{
    IV - elaborar seus orçamentos anuais e plurianuais;
  }

  \only<6>{
    V - adotar regime financeiro e contábil que atenda às suas peculiaridades de
    organização e funcionamento;
  }

  \only<7>{
    VI - realizar operações de crédito ou de financiamento, com aprovação do Poder
    competente, para aquisição de bens imóveis, instalações e equipamentos;
  }

  \only<8>{
    VII - efetuar transferências, quitações e tomar outras providências de ordem
    orçamentária, financeira e patrimonial necessárias ao seu bom desempenho.
  }

  \only<9>{
    § 2º Atribuições de autonomia universitária poderão ser estendidas a
    instituições que comprovem alta qualificação para o ensino ou para a pesquisa,
    com base em avaliação realizada pelo Poder Público.
  }
\end{frame}

\begin{frame}{}
  Art. 55. Caberá à União assegurar, anualmente, em seu Orçamento Geral,
  recursos suficientes para manutenção e desenvolvimento das instituições de
  educação superior por ela mantidas.
\end{frame}

\begin{frame}{}
  Art. 56. As instituições públicas de educação superior obedecerão ao princípio
  da gestão democrática, assegurada a existência de órgãos colegiados
  deliberativos, de que participarão os segmentos da comunidade institucional,
  local e regional.
\end{frame}

\begin{frame}{Art. 56. As instituições públicas de educação superior…}
  Parágrafo único. Em qualquer caso, os docentes ocuparão setenta por cento dos
  assentos em cada órgão colegiado e comissão, inclusive nos que tratarem da
  elaboração e modificações estatutárias e regimentais, bem como da escolha de
  dirigentes.
\end{frame}

\begin{frame}
  Art. 57. Nas instituições públicas de educação superior, o professor ficará
  obrigado ao mínimo de oito horas semanais de aulas.
\end{frame}

\begin{frame}
  \nocite{*}
  \bibliographystyle{plain}
  \bibliography{Items}
\end{frame}
\end{document}
